\documentclass[12pt]{article}

\usepackage{amsmath}
\usepackage{amssymb}
\usepackage{amsthm}
\usepackage{amsfonts}
\usepackage{proof}
\usepackage[margin=1in]{geometry}
\usepackage{braket}

\usepackage[square]{natbib}
%\usepackage[parfill]{parskip}
\author{David Thibodeau \and Jason Wiener}
\title{COMP 512: Deliverable 1}
\date{\today}

%Declare formatting for theorems, definitions, etc.
\theoremstyle{plain}% default 
\newtheorem{thm}{Theorem}
\newtheorem{lem}[thm]{Lemma} 
\newtheorem{prop}[thm]{Proposition} 
\newtheorem*{cor}{Corollary} 

\theoremstyle{definition} 
\newtheorem{defn}{Definition}[section] 
\newtheorem{conj}{Conjecture}[section] 
\newtheorem{exmp}{Example}[section]
\newtheorem{exer}{Exercise}

\theoremstyle{remark} 
\newtheorem*{rem}{Remark} 
\newtheorem*{note}{Note} 
\newtheorem{case}{Case}

\usepackage{srcltx}
\usepackage{listings}
\lstloadlanguages{Java}
\lstset{language=Java}

\newcommand{\veps}{\varepsilon}
\newcommand{\ra}{\rightarrow}
\newcommand{\da}{\Downarrow}
\newcommand{\mips}{{\ensuremath{\text{MIP}^*}} }
\newcommand{\mipns}{{\ensuremath{\text{MIP}^{\text{ns}}}} }
\newcommand{\z}{\mathbb{Z}}

%
% --------------------------------------------------------------------------- 
%
% Set up listings "literate" keyword stuff (for \lstset below)
%
\newdimen\zzlistingsize
\newdimen\zzlistingsizedefault
 \zzlistingsizedefault=11pt
% \zzlistingsizedefault=10pt
%\zzlistingsizedefault=11pt
\zzlistingsize=\zzlistingsizedefault
\global\def\InsideComment{0}
\newcommand{\Lstbasicstyle}{\fontsize{\zzlistingsize}{1.05\zzlistingsize}\ttfamily}
\newcommand{\keywordFmt}{\fontsize{1.0\zzlistingsize}{1.0\zzlistingsize}\bf}
\newcommand{\smartkeywordFmt}{\if0\InsideComment\keywordFmt\fi}
\newcommand{\commentFmt}{\def\InsideComment{1}\fontsize{0.95\zzlistingsize}{1.0\zzlistingsize}\rmfamily\slshape}

\newcommand{\LST}{\setlistingsize{\zzlistingsizedefault}}

\newlength{\zzlstwidth}
\newcommand{\setlistingsize}[1]{\zzlistingsize=#1%
\settowidth{\zzlstwidth}{{\Lstbasicstyle~}}}
\setlistingsize{\zzlistingsizedefault}
\lstset{literate={->}{{$\rightarrow~$}}2 %
                               {→}{{$\rightarrow~$}}2 %
                               {=>}{{$\Rightarrow~$}}2 %
                               {id}{{{\smartkeywordFmt id}}}1 % 3 $~$
                               {\\}{{$\lambda$}}1 %
                               {\\n}{$\backslash n$}1 %
                               {\\Pi}{{$\Pi$}}1 %
                               {\\psi}{{$\psi$}}1 %
                               {\\gamma}{{$\gamma$}}1 %
                               {FN}{{$\Lambda$}}1 %
                               {<<}{\color{mydblue}}1 %
                               {<<r}{\color{dGreen}}1 %
                               {<*}{\color{dRed}}1 %
                               {<dim}{\color{dimgrey}}1 %
                               {>>}{\color{black}}1 %
                               {>>b}{\color{mydblue}}1 %
                               {phi}{$\phi$}1 %
                               {psi}{$\psi$}1 %
                              % {..}{$\dots$}1 %
               ,
               columns=[l]fullflexible,
               basewidth=\zzlstwidth,
               basicstyle=\Lstbasicstyle,
               keywordstyle=\keywordFmt,
               identifierstyle=\relax,
%               stringstyle=\relax,
               commentstyle=\commentFmt,
               breaklines=true,
               breakatwhitespace=true,   % doesn't do anything (?!)
               mathescape=true,   % interprets $...$ in listing as math mode
%               tabsize=8,
               texcl=false}

\newcommand{\java}[1]{{\lstinline!#1!}} 

\newenvironment{figureone}[1]{%
  \def\deffigurecaption{#1}%
  \begin{figure}[htbp]%
  \begin{center}%
  %\begin{scriptsize}%
  \begin{minipage}{\columnwidth}%
  \hrule \vspace*{2ex}%
%   % reassign dimensions for prooftrees
%   \proofrulebaseline=2ex%
%   \proofrulebreadth=.05em%
%   \proofdotseparation=1.25ex%
}{%
%\vspace{2ex} \hrule% 
%\addvspace{2ex}%
  \end{minipage}%
  %\end{scriptsize}%
  \end{center}%
  \caption{\deffigurecaption}%
  \end{figure}%
}

\begin{document}
\maketitle

%TODO: Put code examples
Our implementation is split into three layers: the client which takes
user inputs and formats them into queries that it sends to the server;
the middleware which receives queries from the client and then redirects and/or
coordinates the execution of the queries with the ressource managers;
the ressource managers which each keeps track of a particular
ressource. There are four ressources: cars, flights, rooms, and
customers. Some operations such as adding a ressource, deleting a
ressource (except for customers) or queries informations about them
only requires the middleware to pass the query to the corresponding
ressource manager. 
\begin{lstlisting}
public boolean addFlight(int id, int flightNum, int flightSeats, int flightPrice) 
    throws RemoteException {
  return rmFlight.addFlight(id, flightNum, flightSeats, flightPrice);
}
\end{lstlisting}
Other operations such as reservations and deletions
of customers require communicating with both the
customer RM and the object RMs.

\subsubsection*{Implementation using Java's RMI}
We have two implementations of the project, one using Java's Remote
Method Invocation while the other handles TCP sockets directly. The
RMI implementation uses Java interfaces to indicate to the remote
program what are the local methods it can use. The middleware
implements the \java{ResourceManager} interface. The resource managers
implement one of \java{RMCar}, \java{RMFlight}, \java{RMHotel}, or
\java{RMCustomer} which all extend the base interface \java{RMBase}.
\java{RMBase} is the interface used by the base class
\java{RMBaseImpl} which as parent class for the implementations of
each of the ressource managers. This class implements the hashtable
and operations that are common to all the RMs like \java{readData},
\java{writeData}, and \java{reserveItem}.

To start the implementation, we first need to start the RMs. They take
as optional input a port number. If none is provided, they assume the
port 1099 is used for the rmiregistry (which needs to be started on
the local machine). The middleware then takes as arguments the name of
the machines running RMCar, RMFlight, RMHotel, and RMCustomer,
respectively. It also takes a port number as an optional argument. For
simplicity, it will take a single port number and will try to reach
all the RMs on this port, in addition to setting himself to listen to
that port. This limitation requires us to start all the rmiregistries
on the same port. This, however, simplifies the parsing of the input
for the middleware. We have had some trouble connecting to a
rmiregistry that was not already started. In particular, it could not
see the RMBase interface even if it was there. It seems to be linked to
having the CLASSPATH variable set when the rmiregistry is started.
Then, the client is started and takes the name of the server running
the middleware together with an optional port number.

\subsubsection*{Implementation using Sockets}
The TCP implementation communicates using the Java's Sockets. The
queries to the middleware or the RMs are done by sending a
\java{Vector} whose first element is a string containing a particular
keyword representing the operation to perform. The other elements in
the vector are the arguments to be provided. The client makes sure the
input provided by the user is correct before making the call. The
middleware can then always assume the rest of the vector is properly
formatted. Similarly, the middleware always sends properly formatted
vectors to the RMs so that they don't have to do such verifications. 

Both the middleware and the RMs create a Socket Server and enter
socket accept loops. Each time something connects to it (either the
client to the middleware, or the middleware to one of the RMs), the
server will create an object of its own class with the newly created
socket as a field. Then, it will create a thread for this object so
that the request is handled while the main thread continue to listen
for new connections. 
\begin{lstlisting}
Socket clientSocket;
ServerSocket connection = new ServerSocket(port);

while (true) {
  clientSocket = connection.accept();
  
  TCPMiddleware obj = new TCPMiddleware(clientSocket, server, port);
  Thread t = new Thread(obj);
  t.start();
}
\end{lstlisting}
This thread will then obtain the input and output
streams from this connection and wait for a vector to be passed to
it. When it happens, it will read it and forward the query to the
appropriate method and wait for an answer from it. Then, it will send
back an answer to the client. In the case of some queries that need
the attention of only one RM, the middleware will just forward the
vector to the appropriate RM. 
\begin{lstlisting}
if (!method.contains("reserve")) {
  if (method.contains("Cars")) {
    try {
      carsOut.writeObject(methodInvocation);
      clientOut.writeObject(carsIn.readObject());
    } catch (IOException e) {
      Trace.error("IOException in method invocation: "
      + getString(method));
    }
  }
\end{lstlisting}
This implementation is then non-blocking but will
not scale well to a large number of clients as the host might will
most likely limit the possible number of threads a process can run.
%Discuss that last sentence with Jason. Need also specify how it is
%run.

\subsubsection*{Reservations}
The functions interacting with reservations, such as
\java{reserveCar}, \java{reserveFlight}, \java{reserveRoom},
\java{itinerary}, but also \java{deleteCustomer} which deletes all
reservations from the customer to be deleted need to be handled by the
middleware that needs to update the customer informations and then the
RM dealing with the object to be reserved. In order to avoid
duplicating code, we use an \java{enum} called \java{rType} which
indicates if the object to be reserved is a car, a flight seat or a
room. The reserve functions will 
first verify the existence of the customer. Then, it will call the
corresponding RM to verify if the item can be reserved and do it
altogether. In order to prevent concurrent execution to cause
problems, we lock the object that is being updated as it is retrieved
from the hashtable. Then, it will get the information back and add the
reservation to the customer in its RM. Since another client could have
deleted the customer during that time, it does the check and, if it
passes, adds the reservation to the customer's reservations table. If
the customer has been deleted, the middleware will query the RM of the
reserved object to unreserve it. 
The code appears in figure~\ref{fig:reserve}. 
\begin{figureone}{\java{reserveItem} method from the middleware. \label{fig:reserve}}
\begin{lstlisting}
protected ReservedItem reserveItem(int id, int customerID, String key, 
                                      String location, ReservedItem.rType rtype)
    throws RemoteException {
  // Verifies if customer exists
  Customer cust = rmCustomer.getCustomer(id, customerID);
  if ( cust == null ) {
    return null;
  } 

  RMInteger price = null;
  // check if the item is available
  if (rtype == ReservedItem.rType.CAR)
    price = rmCar.reserveItem(id, customerID, key, location);
  else if (rtype == ReservedItem.rType.FLIGHT)
    price = rmFlight.reserveItem(id, customerID, key, location);
  else if (rtype == ReservedItem.rType.ROOM)
    price = rmHotel.reserveItem(id, customerID, key, location);

  if ( price == null ) {
    return null;
  } else {  
    // We do the following check in case the customer has been
    // deleted between the first verification and now.
    ReservedItem item = rmCustomer.reserve(id, customerID, key,
                                           location, price.getValue(), rtype);
    if(item != null){
      return item;
    } else {
      if (rtype == ReservedItem.rType.CAR)
        rmCar.unreserveItem(id, item);
      else if (rtype == ReservedItem.rType.FLIGHT)
        rmFlight.unreserveItem(id, item);
      else if (rtype == ReservedItem.rType.ROOM)
        rmHotel.unreserveItem(id, item);
    }
    return null;
  }
}

\end{lstlisting}
\end{figureone}

We needed to do it this way since we
could not synchronize the whole reservation as it involved three
differents servers. Similarly, the \java{itinerary} functions does all the
reservations one after the others but as soon as one fails, it undoes
all the previous ones it did. The idea is that the whole itinerary
must be valid or nothing is done. It would make little sense for the
customer to have only partial itineraries. In order to do that, we
made so that \java{reserveItem} returns a \java{ReservedItem} instead
of a boolean so that one could easily unreserve it. We test
success of the reservation by verifying if the \java{ReservedItem} is \java{null}.

 \java{deleteCustomer} will delete the
 customer and then receive from the customer RM the hashtable of
 reservations. Then, it will go through all the reservations and
 unreserve them. We note that an object RM will never unreserve twice
 the same object. First, the only two ways an object reservation is
 cancelled is when the customer is deleted or when a reservation cannot
 go through. If the reservation does not go through because the
 customer is deleted (e.g. if it happens in \java{reserveCar}), then it
 will never be placed in the customer's hashtable of reservations and
 so will not be unreserved by the \java{deleteCustomer}. If the
 reservation is cancelled because the whole itinerary cannot be
 reserved, then each of them will have to lock the customer. If the
 delete function has the lock first, it will set the customer field
 \java{m\_deleted} to \java{true} and so unreserve will return \java{false},
 cancelling the reservation on the RM of the object. If the unreserve
 has the lock first, then it will remove the reservation from the
 customer's hashtable of reservation before \java{deleteCustomer}
 retrieves the hashtable. Hence, \java{deleteCustomer} will never see
 the reservation and will not be able to cancel it. Thus, any
 reservation will be cancelled at most once.

We also put locks on items we reserve in \java{RMBaseImpl}. Without
them, two customers could reserve the same last car, room, or flight
seat. Concurrent reservations and unreservations could also lead to
wrong assignments to the item's \java{nCount} or \java{nReserved}
fields. Moreover, \java{deleteItem} acquires the lock to make sure a
reservation will not be done on a deleted object.

\begin{lstlisting}
public boolean unreserveItem(int id, String key)
    throws RemoteException{
  ReservableItem item = (ReservableItem) readData(id, key);
  synchronized (item) {
    item.setReserved(item.getReserved()-1);
    item.setCount(item.getCount()+1);
    return true;
  }
}
\end{lstlisting}
\subsubsection*{Tests}
We first did simple connection tests to make sure the implementations
actually run. Then, we tested if the functions executed properly and
if the expected answer was given. The simple tests included creating,
reserving and deleting items. Deleting customers, creating
itineraries. The analysis performed above eliminated the need to test
for concurrent reservations and deletions. 

We would have liked but did not have time to test the number of
concurrent clients each of the two implementations can handle.
%Finish tests section
\end{document}